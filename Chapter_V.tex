Unless otherwise stated, all rings in this chapter are \emph{commutative}.

\section{}


\begin{problem}{V.1.1}
Let $R$ be an Notherian ring, and let $I$ be an ideal of $R$. Prove that $R/I$ is a Notherian ring.
\end{problem}
\begin{pf}
The projection $\varphi: R \to R/I$ is clearly an surjective homomorphism, and by III.4.2 $R/I$ is Notherian.
\end{pf}

\begin{problem}{V.1.2}
Prove that if $R[x]$ is Notherian, then so is $R$.
\end{problem}
\begin{pf}
\[
\pi: R[x] \to R[x]/(x) \cong R
\]
is surjective, and by V.1.1 $R$ is Notherian.
\end{pf}

\begin{problem}{V.1.4}
Let $R$ be the ring of real-valued continuous functions on the interval $[0,1]$. Prove that $R$ is not Noetherian.
\end{problem}
\begin{pf}
Let $\{f_n\}_{n = 1}^\infty$ be continuous functions so that $f_n$ has support on $[0,1-2^{-n}]$ (i.e. $f_n = 0$ on $(1-2^{-n},1]$). Then  
\[
(f_1) \subseteq (f_2) \subseteq \cdots (f_n) \subseteq \cdots	
\] 
is an increasing sequence of ideals that does not terminate. By Proposition V.1.1, $R$ is not Noetherian.
\end{pf}

\begin{problem}{V.1.6}
Let $I$ be an ideal of $R[x]$, and let $A \subseteq R$ be the set defined in the proof of Theorem 1.2. Prove that $A$ is an ideal of $R$.
\end{problem}
\begin{pf}
$A$ is a subgroup of $(R,+)$: For $a, b \in A$, there is some $f, g \in I$ so that the leading coefficient of $f$ (resp. $g$) is $a$ (resp. $b$). Assume that $\deg(f) \geq \deg(g)$. Then $f - x^{\deg(f)-\deg(g)}g$ is an element of $I$, and it has leading coefficient $a-b$, which is in $A$, so $A$ is a subgroup.

$A$ satisfies absorption property: If $a \in R$, then there is some $f \in I$ such that $a$ is the leading coefficient of $f$. Then $rf \in I$ has leading coefficient $ra$, which is in $A$, so $ra \in A$ for all $r \in R$. Therefore $A$ is an ideal.   
\end{pf}

\begin{problem}{V.1.8}
Prove that every ideal in a Noetherian ring $R$ contains a finite product of prime ideals. 
\end{problem}
\begin{pf}
Suppose there are some ideals that does not contain a finite product of prime ideals. Let us collect these ideals and form a family $\mathscr{F}$, which clearly is nonempty. Since $R$ is Noetherian, there is an maximal ideal with respect to inclusion in $\mathscr{F}$, which we call it $M$. Since $M$ is not prime, there exists $a,b \notin M$ such that $ab \in M$. Now consider two ideals that are larger than $M$ (so they contain a finite product of prime ideals):
\[
M + (a), M + (b)
\]
Note that both of them are \emph{proper}: If $M+aR = R$, then $bM + baR = bR$, and since $bM + baR \subseteq M$ we would have $bR \subseteq M$, i.e. $b \in M$, a contradiction. Then since
\[
(M+(a))(M+(b)) \subseteq M
\]
and since the product on the left contains a finite product of prime ideals, $M$ contains a finite product of prime ideals, a contradiction. Therefore $\mathscr{F} = \varnothing$, and the assertion is proved.
\end{pf}

\begin{problem}{V.1.12}
Let $R$ be an integral domain. Prove that a nonzero $a$ is irreducible if and only if $(a)$ is maximal among proper principle ideal of $R$.
\end{problem}
\begin{pf}

\noindent $(\Rightarrow)$ If $a$ is irreducible but there is some $b \in R$ such that $(a) \subseteq (b)$, then we can write $a = bc$ for some $c \in R$. Then either $b$ is a unit, or $c$ is a unit. The former would lead to that $(b) = R$, and the latter says that there is also $c^{-1}$ such that $ac^{-1} = b$, so $(a)\supseteq (b)$, so $(a) = (b)$. Either way, $(a)$ is the maximal amongst all principle ideals. \\
$(\Leftarrow)$ If $a = bc$, then $(a) \subseteq (b)$. Since $(a)$ is maximal amongst all principle ideal, we must have $(b) = R$ or $(b) = (a)$. In the former we have that $b$ is a unit, and the latter implies that $c$ is a unit. In both cases at least one of $b$ and $c$ is a unit, so $a$ is irreducible. 
\end{pf}

\begin{problem}{V.1.17}
    Consider the subring of $\C$:
    \[
    \Z[\sqrt{-5}] := \{a+bi\sqrt{5} \; | \; a,b\in Z\}	
    \]
    Prove that $\Z[\sqrt{-5}]$ is not a UFD.
    \end{problem}
    \begin{pf}
\begin{itemize}
    \setlength\itemsep{0pt}
    \item By the same argument as in the 4th point of III.4.10, $\Z[\sqrt{-5}] \cong \Z[t]/(t^2-5)$.
    \item $\Z[t]$ is Noetherian ($\Z$ is Noetherian and Hilbert Basis), so $\Z[t]/(t^2-5) \cong \Z[\sqrt{-5}]$ is Noetherian by V.1.1. Since $(t^2 + 5)$ is maximal (hence prime), the quotient $\Z[t]/(t^2-5) \cong \Z[\sqrt{-5}]$ is a domain.
    \item The norm $N(a+bi\sqrt{5}) = a^2 + 5b^2$ satisfies the multiplicative property by the same argument as in the 2nd point of III.4.10.
    \item If an element $u$ is a unit, then we must have $N(u)N(u^{-1}) = N(1) = 1$, and this forces $N(u) = 1$ as the definition of norm guarantees $N(a) \geq 1$ for all nonzero $a$, so $u = \pm 1 $.
    \item If $a, b \in \Z[\sqrt{-5}]$ satisfies $ab = 2$ (resp. $3, 1+i\sqrt{5}, 1-i\sqrt{5}$), then we have $N(a)N(b) = 4$ (resp. $9,6,6$). If $N(a) \geq N(b)$, then we must have $N(a) = 4$ (resp. $9,6,6$) since $\Z[\sqrt{-5}]$ does not contain elements such that $N(a) = 2 \text{ or } 3$.
    \item $6 = 2 \cdot 3 = (1 + i \sqrt{5}) (1 - i \sqrt{5})$.
    \item Since the factorization of $6$ is not unique, $\Z[\sqrt{-5}]$ is not a UFD.
\end{itemize}
\end{pf}

\section{}

\begin{problem}{V.2.1}
Prove Lemma 2.1: 

\textit{
Let R be a UFD, and let a,b,c be nonzero elements of R. Then
\begin{itemize}
    \setlength\itemsep{0pt}
    \item $(a)\subseteq(b) \Longleftrightarrow$ the multiset of irreducible factors of b is contained in the multiset of irreducible factors of a;
    \item a and b are associates $\Longleftrightarrow$ the two multiset coincide;
    \item the irreducible factors of a product bc are the collection of all irreducible factors of b and c. 
\end{itemize}
}
\end{problem}
\begin{pf}
\begin{itemize}	
    \setlength\itemsep{0pt}
    \item The inclusion on the right implies $a = bp$ for some $p$, so clearly the multiset of $a$ contains the multiset of $b$. Conversely, we can let $p$ be the product of difference of the multiset of $a$ and $b$. Then $a = pb$ (up to associates), so $(a)\subseteq(b)$.
    \item A unit $u$ is \emph{not} a product of irreducibles by definition. Therefore if $(a) = (b)$, then $a = bn$ for some unit $n$, and since $n$ does not contain irreducibles, the multiset must coincide. The converse is just the reverse of this argument.
    \item Direct by expanding $b$ and $c$.
\end{itemize}
\end{pf}

\begin{problem}{V.2.5}
Let $R$ be the subring of $\Z[t]$ consisting of polynomials with no term of degree 1.
\begin{itemize}
    \setlength\itemsep{0pt}
    \item Prove that $R$ is indeed a subring of $\Z[t]$, and conclude that $R$ is an integral domain.
    \item List all common divisor of $t^5$ and $t^6$ in $R$.
    \item Prove that $t^5$ and $t^6$ have no gcd in $R$.
\end{itemize}
\end{problem}
\begin{pf}
\begin{itemize}
    \setlength\itemsep{0pt}
    \item Clearly $R$ is a subring since the difference of two polynomials in $R$ has no term of degree 1. It is a domain since you still can't have two nonzero polynomials that has product $0$.
    \item If $(t^5, t^6) \subseteq (p)$, then $p$ can be $1, t, t^2, t^3, t^4$ or $t^5$.
    \item gcd did not exist since for any $k \in \{0,1,2,3,4,5\}$, $t^{k-1} \mid t^5, t^{k-1} \mid t^6$, but $t^{k-1} \nmid t^k$ since $R$ does not contain $t$.
\end{itemize}
\end{pf}